\section{Introduction}
\label{section: introduction}

\subsection{Aim of the project}
The idea is to find a cost-performance ratio comparing methods (and hopefully in
hardware) like \textit{YOLO}, \textit{SSD}, \textit{R-CNN} and
\textit{Fast R-CNN} for object detection, this focused mainly on autonomous
driving, and thus find a way to implement autonomous driving at lower costs
without losing much accuracy.

\subsection{Object Detection for Autonomous Driving}
Is Autonomous driving only a \textit{"Sci-fi dream"}? Or can we make it real?\\

Well, there have been a lot of projects and researchs to make this possible, and
little by little we are getting cloese to it. But why bother to do autonomous
driving? Is there any benefit? We have some benefits like \textbf{safety},
\textbf{mobility}, \textbf{environmental}, \textbf{efficiency}, even
\textbf{economic} ones, a NHTSA
(\textit{National Highway Traffic Safety Administration}) study showed that
motor vehicle crashes cost billions each year. Eliminating the majority of
vehicle crashes through technology could reduce this cost \cite{nhtsa}.\\
moment of deciding
The future is a \textbf{Road to Full Automation}. Cars and trucks that drive us,
instead of us driving them may offer transformative safety opportunities at
their maturity. At this time, even the highest level of driving automation
available to consumers requires the full engagement and undivided attention of
drivers. There is considerable investment into safe testing, development and
validation of automated driving systems. These automotive technology
advancements also have the potential to improve equity, air pollution,
accessibility and traffic congestion. \\

Therefore, in recent years the topic of automatic driving has become popular,
where there is no driver at the wheel, but rather a computer drives internally,
deciding what actions to take regarding what is in front of it. This, as we
assume, is a problem of detection and classification of objects in a situation
of great danger, where if there is an error, a life would be at risk. Therefore,
different techniques are applied depending on what you want to detect, whether
they are pedestrians, cars, traffic signals. And each object to classify has its
own complexity, since it would be more difficult to detect a person from a large
group of people, but when driving, does the detection of one or more people
change the decision of my robotic driver? Each author could propose a different
approach, since the idea encompasses countless problems and techniques used to
solve these problems. \\

Object detection algorithms based on deep learning are mainly used in vehicle
detection, pedestrian detection and traffic sign detection in autonomous driving
scenarios, so depending on what is to be detected, a technique is used. or
another, depending on the precision, speed or complexity that you need at the
decision making.